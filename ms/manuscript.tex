\documentclass{article}
\usepackage[utf8]{inputenc}
\usepackage[margin=2cm]{geometry}
\usepackage{amsmath, amssymb}
\usepackage{bm}
\usepackage{gensymb}
\usepackage{lineno}
\usepackage{lscape}
\renewcommand\linenumberfont{\normalfont\bfseries\small\color{darkgrey}}
\usepackage{booktabs}
\usepackage[round]{natbib}
\bibliographystyle{plainnat}
\usepackage{authblk}
% Linux Libertine:
\usepackage{textcomp}
\usepackage[sb]{libertine}
\usepackage[varqu,varl]{inconsolata}% sans serif typewriter
\usepackage[libertine,bigdelims,vvarbb]{newtxmath} % bb from STIX
\usepackage[cal=boondoxo]{mathalfa} % mathcal
\useosf % osf for text, not math
\usepackage[supstfm=libertinesups,%
  supscaled=1.2,%
  raised=-.13em]{superiors}
\usepackage{setspace}
\usepackage{siunitx}
\usepackage[section]{placeins} % Keep floats (figs, tabs, eqns) in the right section
% I assume these all work together to make nice colour
\usepackage[dvipsnames]{xcolor}
\definecolor{niceblue}{HTML}{236899} % Depends \usepackage[dvipsnames]{xcolor}
\definecolor{darkgrey}{HTML}{A9A9A9} % Depends \usepackage[dvipsnames]{xcolor}
\usepackage{hyperref}
\hypersetup{
    colorlinks=true,
    linkcolor=black,
    filecolor=black,      
    urlcolor=niceblue,
    citecolor=niceblue,
    linkbordercolor = white
}
% End nice colour
\usepackage{pgfplotstable} % For tables
\pgfplotsset{compat=1.16} % For tables
\newcommand{\yearStart}{1979}
\newcommand{\yearEnd}{2017}
\newcommand{\nYears}{39}
\newcommand{\nRegionsThree}{3}
\newcommand{\nRegionsSix}{6}
\newcommand{\nRegionsEight}{8}
\newcommand{\nReleased}{880632}
\newcommand{\nReleasedAK}{357502}
\newcommand{\nReleasedBC}{485261}
\newcommand{\nReleasedCC}{37869}
\newcommand{\nRecovered}{61713}
\newcommand{\nRecoveredAK}{16835}
\newcommand{\nRecoveredBC}{42456}
\newcommand{\nRecoveredCC}{2422}
\newcommand{\nReleasedRaw}{944741}
\newcommand{\nReleasedAKRaw}{380539}
\newcommand{\nReleasedBCRaw}{524719}
\newcommand{\nReleasedCCRaw}{39483}
\newcommand{\nRecoveredRaw}{114267}
\newcommand{\nRecoveredAKRaw}{39803}
\newcommand{\nRecoveredBCRaw}{69737}
\newcommand{\nRecoveredCCRaw}{4727}
\newcommand{\daysDurationMinRaw}{1}
\newcommand{\daysDurationMeanRaw}{1223}
\newcommand{\daysDurationMaxRaw}{13581}
\newcommand{\distanceMinRaw}{0}
\newcommand{\distanceMeanRaw}{346}
\newcommand{\distanceMaxRaw}{4806}
\newcommand{\daysDurationMin}{90}
\newcommand{\releasedSizeMin}{400}
\newcommand{\releasedSizeMax}{800}
\newcommand{\releasedSizeSmallMin}{400}
\newcommand{\releasedSizeSmallMax}{549}
\newcommand{\releasedSizeLargeMin}{550}
\newcommand{\releasedSizeLargeMax}{800}
\newcommand{\muNaturalMortality}{0.1}
\newcommand{\sdNaturalMortality}{0.01}
\newcommand{\muInitialLossRate}{0.1}
\newcommand{\sdInitialLossRate}{0.01}
\newcommand{\muOngoingLossRate}{0.02}
\newcommand{\sdOngoingLossRate}{0.001}
\newcommand{\muReportingRateAK}{0.4}
\newcommand{\muReportingRateBC}{0.5}
\newcommand{\muReportingRateCC}{0.3}
\newcommand{\sdReportingRateAK}{0.04}
\newcommand{\sdReportingRateBC}{0.05}
\newcommand{\sdReportingRateCC}{0.03}
\newcommand{\nChains}{1}
\newcommand{\stepSize}{0.01}
\newcommand{\adaptDelta}{0.95}
\newcommand{\iterWarmup}{250}
\newcommand{\iterSampling}{1000}
\newcommand{\maxTreedepth}{10}
\newcommand{\threadsPerChain}{5}

\usepackage{graphicx}
\graphicspath{ {./figs/} }

% Title
\title{Sablefish movement and abundance exchange in the northeast Pacific: insights from four decades of tagging}

% Authors
\author[1]{Luke A. Rogers}
\author[]{...}

%Affiliations
\affil[1]{Pacific Biological Station, Fisheries and Oceans Canada, Nanaimo, BC, V9T 6N7, Canada}

\begin{document}

\maketitle
\linenumbers
\setcounter{secnumdepth}{0}

\section{Co-authorship}
Co-authorship and final author order to folow the PSTAT Data Sharing and Collaboration Agreement (\href{https://docs.google.com/document/d/1AXIhq6lO_qOPf7q67s_SiDOD6qEfih0COtxQZo7v-Hc/edit?usp=sharing}{here}).

\section{Abstract}

\section{Introduction}

\section{Methods}

\subsection{Data}

\subsection{Movement model}

\subsection{Model fitting}

\subsection{Abundance exchange}

\subsection{Sensitivity analyses}

\section{Results}

\subsection{Movement rates}

\subsection{Abundance exchange}

\subsection{Sensitivity analyses}

\section{Discussion}

\section{Acknowledgements}

\section{Figures}

% map-regions-6-network
\begin{figure}[htb]
    \centering
    \includegraphics[width = 0.7\textwidth]{map-regions-6-network}
    \caption{Annual sablefish movement rates among six biogeographic regions in the northeast Pacific Ocean. Shown are 1\degree{} (between neighbouring regions: black arrows) percentage movement rates (per fish per year). WAK: West Alaska; EAK: East Alaska; NBC: North British Columbia; SBC: South British Columbia; NCC: North California Current; SCC: South California Current.}
    \label{fig:map-network-regions-6}
\end{figure}

\begin{figure}[htb]
    \centering
    \includegraphics[width = 0.4\textwidth]{map-regions-3-released-recovered}
    \caption{Tagged sablefish released (top panel; coastwide) and recovered (bottom three panels by jurisdiction released) during 1979--2018. AK: Alaska; BC: British Columbia; CC: California Current.}
    \label{fig:map-regions-3-released-recovered}
\end{figure}

\begin{figure}[htb]
    \centering
    \includegraphics[width = 0.7\textwidth]{bar-regions-3-released-by-size}
    \caption{Sablefish released length (fork length: mm) by jurisdiction. AK: Alaska; BC: British Columbia; CC: California Current. Vertical lines show minimum and maximum sablefish length included in the study (solid lines; 400--800 mm) and division between small and large sablefish size classes (dashed line; 550 mm).}
    \label{fig:bar-regions-3-released-by-size}
\end{figure}

\begin{figure}[htb]
    \centering
    \includegraphics[width = 0.7\textwidth]{bar-abundance-exchange}
    \caption{Sablefish abundance exchange (millions; 90\%{} CI) between jurisdictions estimated from annual 1\degree{} movement rates with uncertainty and numbers of fish (age 2+) with uncertainty. Annual movement rates were estimated by year block: 1979--1994; 1995--2006; 2007--2017. Top panel: sablefish movement from BC to AK (above horizontal axis) and from AK to BC (below horizontal axis). Bottom panel: sablefish movement from CC to BC (above horizontal axis) and from BC to CC (below horizontal axis). AK: Alaska; BC: British Columbia; CC: California Current.}
    \label{fig:bar-abundance-exchange}
\end{figure}

\begin{figure}[htb]
    \centering
    \includegraphics[width = 0.7\textwidth]{bar-percent-attributable}
    \caption{Proportion of sablefish abundance (90\%{} CI) within jurisdiction attributable to 1\degree{} movement between jurisdictions. Estimated from annual numbers of fish (age 2+) with uncertainty and annual movement rates with uncertainty. Annual movement rates were estimated by year block: 1979--1994; 1995--2006; 2007--2017. Top panel: proportion of numbers of sablefish in AK attributable to movement from BC (above horizontal axis) and proportion in BC attributable to movement from AK (below horizontal axis). Bottom panel: proportion of numbers of sablefish in BC attributable to movement from CC (above horizontal axis) and proportion in CC attributable to movement from BC (below horizontal axis). AK: Alaska; BC: British Columbia; CC: California Current.}
    \label{fig:bar-percent-attributable}
\end{figure}

\section{Tables}


\begin{table}[h]
  \begin{center}
  \caption{Sablefish mean annual movement rates (per fish per year; 90\%{} CI) between six biogeographic regions in the northeast Pacific Ocean. WAK: West Alaska; EAK: East Alaska; NBC: North British Columbia; SBC: South British Columbia; NCC: North California Current; SCC: South California Current.}
  \label{tab:movement-rate-regions-6-mean}
    \pgfplotstabletypeset[
      col sep=comma,                       % Commas separate .csv values
      columns/{0}/.style={column name={}}, % Replace autofilled column name by " "
      string type                          % Expect characters
    ]{tabs/movement-rate-regions-6-mean.csv}
  \end{center}
\end{table}

\section{Supplementary information}

\subsection{Supplementary figures}

\begin{figure}[htb]
    \centering
    \includegraphics[width = 0.4\textwidth]{bar-regions-3-released-by-year}
    \caption{Number of tagged sablefish released by jurisdiction by year. AK: Alaska; BC: British Columbia; CC: California Current.}
    \label{fig:bar-regions-3-released-by-year}
\end{figure}

\begin{figure}[htb]
    \centering
    \includegraphics[width = 0.7\textwidth]{bar-regions-3-recovered-by-year}
    \caption{Number of tagged sablefish recovered by jurisdiction released (panel rows) and jurisdiction recovered (panel columns) by year. Green: recovered within 3 years of release; Orange: recovered more than three years after release. AK: Alaska; BC: British Columbia; CC: California Current.}
    \label{fig:bar-regions-3-recovered-by-year}
\end{figure}

\begin{figure}[htb]
    \centering
    \includegraphics[width = 0.7\textwidth]{bar-regions-3-duration-at-liberty}
    \caption{Number of tagged sablefish recovered by jurisdiction released (panel rows) and duration at liberty (days) between date released and recovered. Dashed line: 3 years duration at liberty. AK: Alaska; BC: British Columbia; CC: California Current.}
    \label{fig:bar-regions-3-duration-at-liberty}
\end{figure}

\begin{figure}[htb]
    \centering
    \includegraphics[width = 0.7\textwidth]{bar-regions-3-fishing-priors-posteriors}
    \caption{TBD}
    \label{fig:bar-regions-3-fishing-priors-posteriors}
\end{figure}

\begin{figure}[htb]
    \centering
    \includegraphics[width = 0.4\textwidth]{bar-regions-3-mortality-priors-posteriors}
    \caption{TBD}
    \label{fig:bar-regions-3-mortality-priors-posteriors}
\end{figure}

\begin{figure}[htb]
    \centering
    \includegraphics[width = 0.4\textwidth]{bar-regions-3-reporting-priors-posteriors}
    \caption{TBD}
    \label{fig:bar-regions-3-reporting-priors-posteriors}
\end{figure}

\begin{figure}[htb]
    \centering
    \includegraphics[width = 0.7\textwidth]{bar-regions-3-size}
    \caption{TBD}
    \label{fig:bar-regions-3-size}
\end{figure}

\begin{figure}[htb]
    \centering
    \includegraphics[width = 0.7\textwidth]{bar-regions-3-size-no-duration-constraint}
    \caption{TBD}
    \label{fig:bar-regions-3-size-no-duration-constraint}
\end{figure}

\begin{figure}[htb]
    \centering
    \includegraphics[width = 0.7\textwidth]{bar-regions-3-size-no-recovery-transition}
    \caption{TBD}
    \label{fig:bar-regions-3-size-no-recovery-transition}
\end{figure}

\begin{figure}[htb]
    \centering
    \includegraphics[width = 0.7\textwidth]{map-regions-6}
    \caption{TBD}
    \label{fig:map-regions-6}
\end{figure}

\begin{figure}[htb]
    \centering
    \includegraphics[width = 0.7\textwidth]{map-regions-8}
    \caption{TBD}
    \label{fig:map-regions-8}
\end{figure}


\subsection{Supplementary tables}

\begin{table}[h]
  \begin{center}
  \caption{Sablefish movement rates between regions (per fish per year).}
  \label{tab:movement-rate-regions-3-mean}
    \pgfplotstabletypeset[
      col sep=comma,                       % Commas separate .csv values
      columns/{0}/.style={column name={}}, % Replace autofilled column name by " "
      string type                          % Expect characters
    ]{tabs/movement-rate-regions-3-mean.csv}
  \end{center}
\end{table}

\begin{landscape}
\begin{table}[h]
  \begin{center}
  \caption{Sablefish movement rates between regions (per fish per year).}
  \label{tab:movement-rate-regions-8-mean}
    \pgfplotstabletypeset[
      col sep=comma,                       % Commas separate .csv values
      columns/{0}/.style={column name={}}, % Replace autofilled column name by " "
      string type                          % Expect characters
    ]{tabs/movement-rate-regions-8-mean.csv}
  \end{center}
\end{table}
\end{landscape}

\begin{table}[h]
  \begin{center}
  \caption{Sablefish movement rates between regions (per fish per year).}
  \label{tab:movement-rate-regions-3-mean-3x-cv-fishing-rate}
    \pgfplotstabletypeset[
      col sep=comma,                       % Commas separate .csv values
      columns/{0}/.style={column name={}}, % Replace autofilled column name by " "
      string type                          % Expect characters
    ]{tabs/movement-rate-regions-3-mean-3x-cv-fishing-rate.csv}
  \end{center}
\end{table}

\begin{table}[h]
  \begin{center}
  \caption{Sablefish movement rates between regions (per fish per year).}
  \label{tab:movement-rate-regions-3-mean-3x-sd-reporting-rate}
    \pgfplotstabletypeset[
      col sep=comma,                       % Commas separate .csv values
      columns/{0}/.style={column name={}}, % Replace autofilled column name by " "
      string type                          % Expect characters
    ]{tabs/movement-rate-regions-3-mean-3x-sd-reporting-rate.csv}
  \end{center}
\end{table}

\begin{table}[h]
  \begin{center}
  \caption{Sablefish movement rates between regions (per fish per year).}
  \label{tab:movement-rate-regions-3-mean-block-1979-1994}
    \pgfplotstabletypeset[
      col sep=comma,                       % Commas separate .csv values
      columns/{0}/.style={column name={}}, % Replace autofilled column name by " "
      string type                          % Expect characters
    ]{tabs/movement-rate-regions-3-mean-block-1979-1994.csv}
  \end{center}
\end{table}

\begin{table}[h]
  \begin{center}
  \caption{Sablefish movement rates between regions (per fish per year).}
  \label{tab:movement-rate-regions-3-mean-block-1995-2006}
    \pgfplotstabletypeset[
      col sep=comma,                       % Commas separate .csv values
      columns/{0}/.style={column name={}}, % Replace autofilled column name by " "
      string type                          % Expect characters
    ]{tabs/movement-rate-regions-3-mean-block-1995-2006.csv}
  \end{center}
\end{table}

\begin{table}[h]
  \begin{center}
  \caption{Sablefish movement rates between regions (per fish per year).}
  \label{tab:movement-rate-regions-3-mean-block-2007-2017}
    \pgfplotstabletypeset[
      col sep=comma,                       % Commas separate .csv values
      columns/{0}/.style={column name={}}, % Replace autofilled column name by " "
      string type                          % Expect characters
    ]{tabs/movement-rate-regions-3-mean-block-2007-2017.csv}
  \end{center}
\end{table}


\newpage
\bibliography{refs}

% End
\end{document}







