\documentclass{article}
\usepackage[utf8]{inputenc}
\usepackage[margin=2cm]{geometry}
\usepackage{amsmath, amssymb}
\usepackage{bm}
\usepackage{gensymb}
\usepackage{lineno}
\usepackage{lscape}
\renewcommand\linenumberfont{\normalfont\bfseries\small\color{darkgrey}}
\usepackage{booktabs}
\usepackage[round]{natbib}
\bibliographystyle{plainnat}
\usepackage{authblk}
% Linux Libertine:
\usepackage{textcomp}
\usepackage[sb]{libertine}
\usepackage[varqu,varl]{inconsolata}% sans serif typewriter
\usepackage[libertine,bigdelims,vvarbb]{newtxmath} % bb from STIX
\usepackage[cal=boondoxo]{mathalfa} % mathcal
\useosf % osf for text, not math
\usepackage{setspace}
\usepackage{siunitx}
\usepackage[section]{placeins} % Keep floats (figs, tabs, eqns) in the right section
% I assume these all work together to make nice colour
\usepackage[dvipsnames]{xcolor}
\definecolor{niceblue}{HTML}{236899} % Depends \usepackage[dvipsnames]{xcolor}
\definecolor{darkgrey}{HTML}{A9A9A9} % Depends \usepackage[dvipsnames]{xcolor}
% End nice colour
\usepackage{pgfplotstable} % For tables
\pgfplotsset{compat=1.16} % For tables
\usepackage{graphicx}
\graphicspath{ {./figs/} }
% hyperref must be last in preamble
\usepackage{hyperref}
\hypersetup{
    colorlinks=true,
    linkcolor=black,
    filecolor=black,      
    urlcolor=niceblue,
    citecolor=niceblue,
    linkbordercolor = white
}

% Title
\title{Sablefish equations }

% Authors
\author[1]{Luke A. Rogers}
\author[]{...}

%Affiliations
\affil[1]{Pacific Biological Station, Fisheries and Oceans Canada, Nanaimo, BC, V9T 6N7, Canada}

\begin{document}

\maketitle
\linenumbers
\setcounter{secnumdepth}{0}

\section{Movement model: decision point}

Briefly: I propose two options for index notation---compact and verbose---and I advocate for compact.
\newline

\noindent Background: The largest model arrays have 5 dimensions that correspond to 5 index slots. Two slots are spatial, two are temporal, and one is size.
\newline

\noindent Details: The compact option encodes the two spatial indexes in one vector $\boldsymbol{s} = \left( s_0,s \right)$ and the two temporal indexes in another vector $\boldsymbol{t} = \left( t_0, t \right)$ wherever possible. This reduces the number of subscript slots displayed in the equations.
\newline

\noindent Tradeoffs: The compact option uses more symbols (vectors, scalars) and faces (bold, plain) in the subscripts, but never uses more than three subscript slots. The verbose option uses fewer symbols and only one face (plain) in the subscripts, but uses up to five subscript slots.
\newline

\noindent Example:
% Example
\begin{align}
  \mathrm{Compact:}\qquad \quad y_{\boldsymbol{s}, \boldsymbol{t},l} &\sim \mathrm{NegBinom2} \!
                                                                 \left[\mu_{\boldsymbol{s}, \boldsymbol{t},l}
                                                                 \mathrm{,} \: \phi \right] \\
  \mathrm{Verbose:}\qquad y_{s_0,s,t_0,t,l} &\sim \mathrm{NegBinom2} \!
                                              \left[\mu_{s_0,s,t_0,t,l} \mathrm{,} \: \phi \right]
\end{align}

\noindent Upshot: I recommend the compact notation because it is concise. I believe it gives a better `sense' of the model on a quick scan, and a more direct route to understanding the nuances on a deep dive.

\newpage

\section{Movement model: compact indexes}

\noindent Indexes: The index notation encodes the two spatial indexes in one vector $\boldsymbol{s} = \left( s_0,s \right)$ and the two temporal indexes in another vector $\boldsymbol{t} = \left( t_0, t \right)$ wherever possible. The components are shown explicitly where they are needed. Shorthands for the initial conditions $\boldsymbol{s}_0 = (s_0,s_0)$ and $\boldsymbol{t}_0 = (t_0,t_0)$ indicate the spatial and temporal indexes corresponding to a tag release event.
\newline

\noindent Symbols: Bold face symbols (other than indexes) correspond to square (sometimes diagonal) matrix slices of larger arrays. Matrix rows correspond to initial or previous spatial regions, while columns correspond to current regions. Consequently, the first two subscript index slots ($\boldsymbol{s} = (s_0, s)$) are suppressed for matrices. Matrix location within an array is given by the values in the remaining subscript index slots. 

Plane face symbols are scalars (often elements of arrays). Matrix symbols are used except where values are inherently univariate, for example the scalar outcome of a univariate sampling distribution. Scalar membership in an array is identified by symbol letter (greek or latin) and letter case. Consequently, different letter cases (e.g. $\boldsymbol{\lambda}_{l}$ and $\boldsymbol{\Lambda}_{t,l}$) correspond to different arrays.

The $\mathrm{diag} \! \left[ \cdot \right]$ function constructs a vector from the diagonal of a square (and diagonal) matrix.


% Sampling model
\begin{equation}
  \label{eq:model-sampling}
  y_{\boldsymbol{s}, \boldsymbol{t},l} \sim \mathrm{NegBinom2} \!
    \left[\mu_{\boldsymbol{s}, \boldsymbol{t},l} \mathrm{,} \: \phi \right]
\end{equation}

% % Observation model
% \begin{equation}
%   \label{eq:model-observation}
%   \widehat{y}^{\:\mathrm{obs}}_{a,b,c,d,e} \sim 
%     \mathrm{Distrn} \! \left[\widehat{y}^{\:\mathrm{proc}}_{a,b,c,d,e} \mathrm{,} \: \mathrm{param} \right]
% \end{equation}

% % Process model
% \begin{equation}
%   \label{eq:model-process}
%   \mu_{\boldsymbol{s}, \boldsymbol{t},l} = 
%     N_{\boldsymbol{s}, \boldsymbol{t},l} \left(1 - \mathrm{exp} \! \left[ - \lambda_{s,l} \,
%     \omega_{s, k[t]} F_{s,\mathrm{year}\left[ t \right]} \right] \right) W_{s}
% \end{equation}

% Process model
\begin{equation}
  \label{eq:model-process}
  \boldsymbol{\mu}_{\boldsymbol{t},l} = \boldsymbol{N}_{\boldsymbol{t},l} \, \boldsymbol{\Psi}_{t,l}
\end{equation}

% % Numbers model
% \begin{equation}
%   \label{eq:model-numbers}
%   \boldsymbol{N}_{t_{0}, t, l} = \boldsymbol{N}_{t_{0}, t-1, l} \:
%     \mathrm{diag} \! \left[ - \mathrm{exp} \! \left[ - \boldsymbol{Z}_{t-1,l} \right]  \right]
%     \boldsymbol{\Gamma}_{\mathrm{block}[t-1],l} \mathrm{,} \quad t > t_0
% \end{equation}

% Numbers model
\begin{equation}
  \label{eq:model-numbers}
  \boldsymbol{N}_{t_{0}, t+1, l} = \boldsymbol{N}_{t_{0}, t, l} \,
    \boldsymbol{\Lambda}_{t,l} \,
    \boldsymbol{\Gamma}_{\mathrm{block}[t],l} \mathrm{,} \quad t \geq t_0
\end{equation}

% Reporting model
\begin{equation}
  \label{eq:model-reporting}
  \boldsymbol{\Psi}_{t,l} = \left( \boldsymbol{I} - \mathrm{exp} \! \left[ -
    \boldsymbol{\lambda}_{l} \boldsymbol{\omega}_{k[t]}  
    \boldsymbol{F}_{\mathrm{year}[t]} \right] \right) \boldsymbol{W}
\end{equation}

% % Reporting model
% \begin{equation}
%   \label{eq:model-reporting}
%   \boldsymbol{\Psi}_{t,l} = \mathrm{diag} \! \left[ \left( 1 - \mathrm{exp} \! \left[ -
%     \boldsymbol{\lambda}_{l} \odot \boldsymbol{\omega}_{k[t]} \odot 
%     \boldsymbol{F}_{\mathrm{year}[t]} \right] \right) \odot \boldsymbol{W} \right]
% \end{equation}

% % Mortality model
% \begin{equation}
%   \label{eq:model-mortality}
%   \boldsymbol{Z}_{t,l} = \boldsymbol{\lambda}_{l} \odot \boldsymbol{\omega}_{k[t]} \odot 
%     \boldsymbol{F}_{\mathrm{year}[t]} + \frac{1}{K} \boldsymbol{M} + \frac{\eta}{K}
% \end{equation}

% % Mortality model
% \begin{equation}
%   \label{eq:model-mortality}
%   Z_{s,t,l} = \lambda_{s,l} \omega_{s,k[t]} 
%     F_{s,\mathrm{year}[t]} + \frac{1}{K} M_{s} + \frac{\eta}{K}
% \end{equation}

% Survival model
\begin{equation}
  \label{eq:model-survival}
    \boldsymbol{\Lambda}_{t,l} = 
    - \mathrm{exp} \! \left[ -
    \boldsymbol{\lambda}_{l} \boldsymbol{\omega}_{k[t]}  
    \boldsymbol{F}_{\mathrm{year}[t]} + \frac{1}{K} \boldsymbol{M} + \frac{\eta}{K} \right]
\end{equation}

% % Survival model
% \begin{equation}
%   \label{eq:model-survival}
%     \boldsymbol{\Lambda}_{t,l} = 
%     \mathrm{diag} \! \left[ - \mathrm{exp} \! \left[ -
%     \boldsymbol{\lambda}_{l} \odot \boldsymbol{\omega}_{k[t]} \odot 
%     \boldsymbol{F}_{\mathrm{year}[t]} + \frac{1}{K} \boldsymbol{M} + \frac{\eta}{K} \right] \right]
% \end{equation}

% Release model
\begin{equation}
  \label{eq:model-release}
  N_{\boldsymbol{s}_{0},\boldsymbol{t}_{0},l} \sim 
    \mathrm{Binom} \! \left[ x_{s_{0},t_{0},l} \mathrm{,} \: \nu \right]
\end{equation}

% Movement model
\begin{equation}
  \label{eq:model-movement}
  \boldsymbol{P}_{\mathrm{year}[t]} = \boldsymbol{\Gamma}^{K}_{\mathrm{year}[t]}
\end{equation}

% % TODO Move to table
% % Seasons model
% \begin{equation}
%   \label{eq:model-seasons}
%   K = 6
% \end{equation}

% % Move to text
% % Variance model
% \begin{equation}
%   \label{eq:model-variance}
%   \mathrm{Var} \! \left[ y  \right] = \mu + \frac{\mu^2}{\phi}
% \end{equation}

\subsection{Priors}

% Prior phi
\begin{equation}
  \label{eq:prior-dispersion}
  \phi \sim \mathrm{Distrn} \! \left[ \mathrm{param} \right]
\end{equation}

% Prior selectivity
\begin{equation}
  \label{eq:prior-selectivity}
  \lambda_{s,l} \sim \mathrm{Beta} \! \left[ \mathrm{val, val} \right]
\end{equation}

% Prior fishing weight
\begin{equation}
  \label{eq:prior-weight}
  \mathrm{diag} \! \left[ \boldsymbol{\omega}_{k[t]} \right] \sim 
    \mathrm{MultiDistrn} \! \left[  \mathrm{val, val} \right]
\end{equation}

% Prior fishing
\begin{equation}
  \label{eq:prior-fishing}
  F_{s,\mathrm{year}} \sim 
    \mathrm{Distrn} \! \left[ F^{\mathrm{assess}}_{s,\mathrm{year}} \mathrm{, val} \right]
\end{equation}

% Prior reporting
\begin{equation}
  \label{eq:prior-reporting}
  W_{s} \sim \mathrm{Beta} \! \left[ \mathrm{val, val} \right]
\end{equation}

% Prior mortality
\begin{equation}
  \label{eq:prior-mortality}
  M_{s}  \sim \mathrm{Distrn} \! \left[ \mathrm{param} \right]
\end{equation}

% Prior tagloss
\begin{equation}
  \label{eq:prior-tagloss}
  \eta \sim \mathrm{Distrn} \! \left[ \mathrm{val, val} \right]
\end{equation}

% Prior initloss
\begin{equation}
  \label{eq:prior-initloss}
  \nu \sim \mathrm{Beta} \! \left[ \mathrm{val, val}  \right]
\end{equation}

\subsection{Tables}

% Model indexes
\begin{table}[ht]
  \centering
  \caption{Movement model indexes}
  \renewcommand\arraystretch{1.2}
  \label{tab:model-indexes}
  \begin{tabular}{l l l l r}
    \toprule
    \textbf{Symbol} & \textbf{Condition} & \textbf{Dimension} & \textbf{Definition} & \textbf{Type} \\
    \toprule
    % Index limits
    $S$ & $= 3$ or $6$ & Scalar & Number of spatial regions & Index limit \\
    $T$ & $\in \mathbb{Z}^{+}$ & Scalar & Number of model time steps & Index limit \\
    $K$ & $= 6$ & Scalar & Number of time steps per year & Index limit \\
    $L$ & $= 1$ or $2$ & Scalar & Number of release size classes & Index limit \\
    \midrule
    % Index variables
    $\boldsymbol{s}$ & $= (s_0, s)$ & Vector & Spatial index vector & Vector index \\
    $\boldsymbol{t}$ & $= (t_0, t)$ & Vector & Temporal index vector & Vector index \\
    $\boldsymbol{s}_0$ & $= (s_0, s_0)$ & Vector & Initial spatial index vector & Vector index \\
    $\boldsymbol{t}_0$ & $= (t_0, t_0)$ & Vector & Initial temporal index vector & Vector index \\
    \midrule
    $s$ & $\in \left[1 \, .. \, S \right]$ & Scalar & Spatial region & Index \\
    $t$ & $\in \left[1 \, .. \, T \right]$ & Scalar & Time step & Index \\
    $s_0$ & $\in \left[1 \, .. \, S \right]$ & Scalar & Initial spatial region & Index \\
    $t_0$ & $\in \left[1 \, .. \, T \! - \! 1 \right]$ & Scalar & Initial model time step & Index \\
    $k$ & $\in \left[1 \, .. \, K \right]$ & Scalar & Time step within year & Index \\
    $l$ & $\in \left[1 \, .. \, L \right]$ & Scalar & Release size class & Index \\
    \bottomrule
  \end{tabular}
\end{table}


\section{Movement model}

\noindent Note: needs updates to language, possibly symbols (C to Q; s to lambda; theta to W?) 

We modeled sablefish movement as a hidden Markov process \cite[][]{langrock-2012-flexible-practical} observed by tag recoveries. Sablefish entered the model as counts of tag releases identified by release region, time-step, and size class. The time-step in our analysis was one quarter-year. For a given release time-step and size class, the initial abundances of tagged sablefish were represented by a square diagonal matrix 

% Survival
\begin{equation}
  \label{eq:abundance-initial}
  \mathrm{diag} \! \left[\boldsymbol{A}_{i,d=1,l}\right] = \boldsymbol{R}_{i,l} \left(1 - \nu \right)
\end{equation}


where row corresponded to release region and diagonal elements were counts of tags released in each region diminished by the initial tag loss rate $\upsilon$. Subsequent tag abundances from the same release time-step and size class were defined recursively as 

% Abundance
\begin{equation}
    \label{eq:abundance}
    \boldsymbol{A}_{i,d,l} = \boldsymbol{A}_{i,d-1,l} \, \boldsymbol{\Gamma}_{n[i,d-1],l}
\end{equation}

\noindent where $n$ was some time-step after release, $\boldsymbol{\Gamma}_n$ acted as a transition matrix, and $\boldsymbol{A}_n$ encoded the projected tag abundances at the beginning of time-step $n$ with rows indicating the release region and columns indicating the destination region. 

What we refer to as the transition matrix $\boldsymbol{\Gamma}_n$ combined survival and movement rates as the matrix product

% Transition
\begin{equation}
    \label{eq:transition}
    \boldsymbol{\Gamma}_{n,l} = \boldsymbol{S}_{n,l} \, \boldsymbol{\Delta}_{l}
\end{equation}

\noindent where $\boldsymbol{S}_t$ was a diagonal matrix of step-wise survival rates in year $t$, $\boldsymbol{\Delta}_t$ was a square matrix of step-wise movement rates in year $t$, and $t[n]$ indexed year as a function of model step. While $\boldsymbol{\Delta}_t$ was a proper right stochastic matrix in the sense that its elements were probabilities and rows summed to one, $\boldsymbol{\Gamma}_n$ was not, because mortality removed some tagged sablefish from the model during each time-step. Consequently, mortality (including tag loss) formed an absorbing Markov state that could not be departed, in addition to the eight discrete non-absorbing states defined by the management regions. We chose to encode the mortality state implicitly via removal for clarity rather than assign an additional row and column to $\boldsymbol{\Gamma}_n$, and we refer to $\boldsymbol{\Gamma}_n$ as a transition matrix for convenience with the understanding that its rows did not sum to one.

Sablefish survival rates depended on mortality and tag loss as

% Survival
\begin{equation}
  \label{eq:survival}
  \mathrm{diag} \! \left[\boldsymbol{S}_{n,l}\right] = 
    \exp\!{\left[-\boldsymbol{s}_l \boldsymbol{f}_n - \boldsymbol{m} - \eta \right]}
\end{equation}

\noindent where $\boldsymbol{f}_n$ and $\boldsymbol{m}$ were row vectors of step-wise fishing and natural mortality, and $\eta$ was step-wise ongoing tag loss including tag-induced mortality. 

We modeled expected tag recoveries by the matrix product of tagged sablefish abundance and a step-wise observation rate. For a given release time-step and size class this was defined as

% Expected
\begin{equation}
  \label{eq:expected}
  \boldsymbol{\widehat{C}}_{i,d,l} = \boldsymbol{A}_{i,d,l} \boldsymbol{B}_{n[i,d],l}
\end{equation}

\noindent where $\boldsymbol{\widehat{C}}_{i,d,l}$ was a square matrix of expected tag recoveries with rows indicating release region and columns indicating recovery region, and $\boldsymbol{B}_{n,l}$ was a diagonal matrix of observation rates defined by the element-wise vector product

% Observation
\begin{equation}
  \label{eq:observation}
  \mathrm{diag} \! \left[\boldsymbol{B}_{n,l}\right] = 
    \boldsymbol{\theta} \left(1 - \exp\!{\left[-\boldsymbol{s}_l \boldsymbol{f}_{n} \right]} \right) 
\end{equation}

\noindent where $\boldsymbol{\theta}$ was a row vector of per-tag reporting rates and $\boldsymbol{f}_n$ was a vector of step-wise fishing mortality rates.

For a given release size class, we defined annual sablefish movement rates by a matrix power of the corresponding step-wise movement rates

% Movement rate
\begin{equation}
    \label{eq:movement}
    \boldsymbol{P}_{l} = \boldsymbol{\Delta}_{l}^K
\end{equation}

\noindent where $\boldsymbol{P}_{l}$ was the square matrix of annual movement rates, and $K$ was the number of time-steps per year.

% The matrix of step-wise movement rate deviations $\boldsymbol{\Lambda}_t = \boldsymbol{\Delta}_t - \boldsymbol{\bar{\Delta}}$ from the step-wise mean movement rates $\boldsymbol{\bar{\Delta}}$ was constrained by an autoregressive process on its diagonal

% Timevary
%\begin{equation}
%  \label{eq:timevary}
%  \mathrm{diag}\!\left[\boldsymbol{\Lambda}_t\right] = \mathrm{diag}\!\left[\boldsymbol{\Lambda}_{t-1}\right]\! \boldsymbol{\Psi} + \boldsymbol{\epsilon}_t, \, \boldsymbol{\epsilon}_t \sim \mathrm{N}\!\left[\boldsymbol{0},\,\boldsymbol{\Sigma}_\Lambda\right]
%\end{equation}

%\noindent where $\boldsymbol{\Psi}$ was a diagonal matrix of autocorrelation coefficients, and $\boldsymbol{\Sigma}_\Lambda$ was a diagonal variance matrix.

We used the NB2 parameterization of the negative binomial distribution with probability mass function

% Sampling
\begin{equation}
  \label{eq:sampling}
  \mathrm{NB2} \! \left[ C_* \mid \widehat{C}_*,\, \phi \right] = \binom{C_* + \phi - 1}{C_*} \left(\frac{\widehat{C}_*}{\widehat{C}_* + \phi}\right)^{C_*} \left( \frac{\phi}{\widehat{C}_* + \phi} \right)^{\phi}
\end{equation}

\noindent where $C_*$ and $\widehat{C}_*$ were the observed and expected tag recovery counts corresponding to a given release region, time-step and length class, and recovery region and time-step.


% Model indexes
\begin{table}[ht]
  \centering
  \caption{Movement model indexes}
  \renewcommand\arraystretch{1.2}
  \label{tab:model-indexes-0}
  \begin{tabular}{l l l l r}
    \toprule
    \textbf{Symbol} & \textbf{Condition} & \textbf{Dimension} & \textbf{Definition} & \textbf{Type} \\
    \toprule
    % Index limits
    $X$ & $\in \mathbb{Z}^{+}$ & Scalar & Number of geographic regions & Index limit \\
    $T$ & $\in \mathbb{Z}^{+}$ & Scalar & Number of years & Index limit \\
    $K$ & $\in \mathbb{Z}^{+}$ & Scalar & Number of time steps per year & Index limit \\
    $N$ & $ = TK $ & Scalar & Number of time steps & Index limit \\
    $D$ & $\in \mathbb{Z}^{+}$ & Scalar & Maximum time steps at liberty & Index limit \\
    $L$ & $\in \mathbb{Z}^{+}$ & Scalar & Number of release size classes & Index limit \\
    \midrule
    % Index variables
    $x, \, y$ & $\in \left[1 \, .. \, X \right]$ & Scalar & Geographic region & Index \\
    $t$ & $\in \left[1 \, .. \, T \right]$ & Scalar & Year & Index \\
    $n$ & $\in \left[1 \, .. \, N \right]$ & Scalar & Time step & Index \\
    $i$ & $\in \left[1 \, .. \, N\!-\!1 \right]$ & Scalar & Release time step & Index \\
    $k$ & $\in \left[1 \, .. \, K \right]$ & Scalar & Time step within year & Index \\
    $d$ & $\in \left[1 \, .. \, D \right]$ & Scalar & Time step at liberty & Index \\
    $l$ & $\in \left[1 \, .. \, S \right]$ & Scalar & Release size class & Index \\
    \bottomrule
  \end{tabular}
\end{table}

% Data
\begin{table}[ht]
  \centering
  \caption{Data}
  \renewcommand\arraystretch{1.2}
  \label{tab:data}
  \begin{tabular}{l l l l r}
    \toprule
    \textbf{Symbol} & \textbf{Condition} & \textbf{Dimension} & \textbf{Definition} & \textbf{Type} \\
    \toprule
    $\bm{R}$ & $\in \mathbb{Z}^{+,0}$ & $[N\!-\!1, L][X]$ & Sablefish tags released (counts) & Stepwise \\
    $\bm{C}$ & $\in \mathbb{Z}^{+,0}$ & $[N\!-\!1, D, L][X, X]$ & Sablefish tags recovered (counts) & Stepwise \\
    \bottomrule
  \end{tabular}
\end{table}

% Model derived quantities
\begin{table}[ht]
  \centering
  \caption{Movement model derived quantities}
  \renewcommand\arraystretch{1.2}
  \label{tab:model-derived}
  \begin{tabular}{l l l l r}
    \toprule
    \textbf{Symbol} & \textbf{Condition} & \textbf{Dimension} & \textbf{Definition} & \textbf{Type} \\
    \toprule
    $\bm{A}$ & $\in \mathbb{R}^{+,0}$ & $[N\!-\!1, D, L][X, X]$ & Sablefish tag abundance (numbers) & Stepwise \\
    $\boldsymbol{\widehat{C}}$ & $\in \mathbb{R}^{+,0}$ & $[N\!-\!1, D, L][X, X]$ & Sablefish expected tags recovered (numbers) & Stepwise \\
    $\boldsymbol{\Gamma}$ & $\in \left[0, 1 \right]$ & $[N, L][X, X]$ & Sablefish transition rates & Stepwise \\
    $\bm{S}$ & $\in \left[0, 1 \right]$ & $[N, L][X, X]$ & Sablefish survival rates & Stepwise \\
    $\bm{B}$ & $\in \left[0, 1 \right]$ & $[N, L][X]$ & Sablefish tag observation rates & Stepwise \\
    \midrule
    $\bm{P}$ & $\in \left[0, 1 \right]$ & $[L][X, X]$ & Sablefish movement rates & Annual \\
    $\bm{F}$ & $\in \mathbb{R}^{+,0}$ & $[T, X]$ & Sablefish instantaneous fishing mortality rates & Annual \\
    $\bm{M}$ & $\in \mathbb{R}^{+,0}$ & $[T, X]$ & Sablefish instantaneous natural mortality rates & Annual \\
%    \midrule
%    $\boldsymbol{\Lambda}$ & $\in \left[0, 1 \right]$ & $[L][X, X]$ & Sablefish movement rate deviations & Stepwise \\    
    \bottomrule
  \end{tabular}
\end{table}


% Model parameters
\begin{table}[ht]
  \centering
  \caption{Movement model parameters}
  \renewcommand\arraystretch{1.2}
  \label{tab:model-parameters-0}
  \begin{tabular}{l l l l r}
    \toprule
    \textbf{Symbol} & \textbf{Condition} & \textbf{Dimension} & \textbf{Definition} & \textbf{Type} \\
    \toprule
    $\boldsymbol{\Delta}$ & $\in \left[0, 1 \right]$ & $[L][X, X]$ & Sablefish movement rates & Stepwise \\
    $\bm{f}$ & $\in \mathbb{R}^{+,0}$ & $[N, X]$ & Sablefish instantaneous fishing mortality rates & Stepwise \\
    $\bm{s}$ & $\in \left[0, 1 \right]$ & $[L, X]$ & Sablefish fisheries size selectivity & Per fish \\
    $\bm{m}$ & $\in \mathbb{R}^{+,0}$ & $[X]$ & Sablefish instantaneous natural mortality rates & Stepwise \\
    $\boldsymbol{\theta}$ & $\in \left[0, 1 \right]$ & $[X]$ & Sablefish tag reporting rates & Per tag \\
    $\eta$ & $\in \mathbb{R}^{+,0}$ & Scalar & Sablefish instantaneous ongoing tag loss rate & Stepwise \\
    $\nu$ & $\in \left[0, 1 \right]$ & Scalar & Sablefish initial tag loss rate & Per tag \\    
    $\phi$ & $\in \mathbb{R}^{+}$ & Scalar & Negative binomial dispersion & NB2 \\
%    \midrule
    \bottomrule
  \end{tabular}
\end{table}




% End
\end{document}

%%% Local Variables:
%%% mode: LaTeX
%%% TeX-master: t
%%% TeX-master: t
%%% TeX-master: t
%%% TeX-master: t
%%% End:
